\setModuleTitle{FASTQC}
\setModuleAuthors{%
    Steve Pederson, Bioinformatics Hub, University of Adelaide \mailto{stephen.pederson@adelaide.edu.au}\\
}
\setModuleContributions{%
  Jimmy Breen, Robinson Research Institute \& Bioinformatics Hub, University of Adelaide \mailto{stephen.pederson@adelaide.edu.au}\\
  Dan Kortschak, Adelson Research Group, University of Adelaide \mailto{dan.kortschak@adelaide.edu.au} \\
  Terry Bertozzi, SA Museum \mailto{Terry.Bertozzi@samuseum.sa.gov.au}
}

\chapter{\moduleTitle}
\newpage

\section{Using fastqc}
\begin{steps}
Removal of low-confidence base calls is an important part of any NGS analysis, and we can begin this process by checking the quality of libraries using the tool \texttt{fastqc}.
As with all programs on the command line, we need to see how it works before we use it.
The following command will open the help file in the \texttt{less} pager which we used earlier.
To navigate through the file, use the \textless spacebar\textgreater ~to move forward a page, \textless \texttt{b}\textgreater ~to move back a page \& \textless \texttt{q}\textgreater ~to exit the manual. \\
\begin{lstlisting}
fastqc -h | less
\end{lstlisting}
\end{steps}

\begin{note}
Fastqc will create an html report on each file you submit, which can be opened from any web browser, such as \texttt{firefox}
As seen in the help page, \texttt{fastqc} can be run from the command line or from a graphic user interface (GUI).
Using a GUI is generally intuitive, so today we will look at the command line usage, as that will give you more flexibility \& options going forward.
Some important options for the command can be seen in the manual.\\
\end{note}
\begin{steps}
As you will see in the manual, setting the \texttt{-h} option as above will call the help page.
Look up the following options to find what they mean. \\
\begin{center}
\begin{tabular}[h]{|p{4cm}|p{8cm}|}
  \hline
  \textbf{Option} & \textbf{Usage} \\
  \hline
  -o & \\
   & \\
   \hline
  -t & \\
   & \\
   \hline
   -{}-casava & \\
   & \\
   \hline
\end{tabular}
\end{center}
\end{steps}

\begin{steps}
Now we have two experiments, so firstly lets run the two paired-end whole-genome sequencing files. First we need to create the output directory, then we can run \texttt{fastqc} using 2 threads which will ensure the files are processed in parallel.
This can be much quicker when dealing with large experiments.
\begin{lstlisting}
cd ~/rawData/WGS
mkdir -p ~/QC/WGS/rawData
fastqc -o ~/QC/WGS/rawData -t 2 WGS_SRR065388_1.fastq.gz WGS_SRR065388_2.fastq.gz
\end{lstlisting}
\end{steps}
It's probably a good idea to scribble a note next to each line if you didn't understand what you did.
If you haven't seen the command \texttt{mkdir} before, check the help page
\begin{lstlisting}
man mkdir
\end{lstlisting}


\begin{steps}
The above command gave both files to fastqc, told it where to write the output (\texttt{-o \~{}/QC}) \& requested two threads (\texttt{-t 2}).
The reports are in the html files, with all of the plots stored in the zip files.
To look at the QC report for each file, we can use \texttt{firefox}.
\begin{lstlisting}
cd ~/QC/RNASeq/rawData
ls -lh
firefox WGS_SRR065388_1_fastqc.html WGS_SRR065388_2_fastqc.html &
\end{lstlisting}
The left hand menu contains a series of click-able links to navigate through the report, with a quick guideline about each section given as a tick, cross of exclamation mark.
\end{steps}

\begin{note}
Two hints which may make your inspection of these files easier are:
\begin{enumerate}
	\item To zoom out in \texttt{firefox} use the shortcut Ctrl-. Reset using Ctrl0 and zoom in using Ctrl+
	\item You can open these directly from a traditional directory view by double clicking on the .html file.
\end{enumerate}
If your terminal seems busy after you close \texttt{firefox}, use the 'Ctrl C' shortcut to stop whatever is keeping it busy
\end{note}

\begin{questions}
How many reads are there in the fastq file?\\
\begin{answer}
  2000000 \\
\end{answer}
How long are the reads in both of the files?\\
\begin{answer}
  100bp \\
\end{answer}
\end{questions}

\section{Interpreting the FASTQC Report}
\begin{note}
As we work through the QC reports we will develop a series of criteria for filtering and cleaning up our files.
There is usually no perfect solution, we just have to make the best decisions we can based on the information we have.
Some sections will prove more informative than others, and some will only be helpful if we are drilling deeply into our data.
Firstly we'll just look at a selection of the plots.
\end{note}

\subsubsection*{Per Base Sequence Quality}
\begin{steps}
Both of the files should be open in \texttt{firefox} in separate tabs.
Perform the following steps on both files.
Click on the \texttt{Per base sequence quality} hyperlink on the left of the page \& you will see a boxplot of the QC score distributions for every position in the read.
This is the main plot that researchers will look at for making informed decisions about later stages of the analysis.
\end{steps}

\begin{questions}
What do you notice about the QC scores as you progress through the read? \\
\begin{answer}
They clearly drop off as the read extends.\\
\end{answer}
\end{questions}

We will deal with trimming the reads in a later section, but start to think about what you should do to these reads to ensure the highest quality in your final alignment \& analysis.

\paragraph{Per Tile Sequence Quality}
This section just gives a quick visualisation about any physical effects on sequence quality due to the tile within the each flowcell.
For the first file, you will notice an even breakdown in the quality of sequences near the end of the reads across all tiles.
In our second QC report, you will notice a poor quality around the 25th base in the 2nd (or 3rd) tile.
Generally, this would only be of note if drilling deeply to remove data from tiles with notable problems.
Most of the time we don't factor in spatial effects, unless alternative approaches fail to address the issues we are dealing with.

\paragraph*{Per Sequence Quality Scores}
This is just the distribution of average quality scores for each sequence.
There's not much of note for us to see here.

\paragraph{Per Base Sequence Content}
This will often show artefacts from barcode sequences or adapters early in the reads, before stabilising to show a relatively even distribution of the bases.

\paragraph{Sequence Duplication Levels}
This plot shows about what you'd expect from an whole genome experiment.
There are a few duplicated sequences lots of unique sequences representing the diverse genome.
While this FastQC report was calculated on two small files (subset from a much larger sequencing run), each FastQC analysis is calculated on a small sample of the library for computational efficiency. However this will give a rough guide if anything unusual stands out

\paragraph{Kmer Content}
Statistically over-represented sequences can be seen here \& often they will overlap.
In our first plot, the some sequences derive from the same motif, and are the extracts of the same longer sequence, just shifted along one base.
No information is given as the source of these sequences, and you would expect to see barcode sequences or motifs that correspond to any digestion protocols here.
This is a plot that can cause significant confusion, but can alert you to any unexpected sequence-based problems in your data.

\subsection{A Different Dataset}
The whole-genome sequencing dataset is relatively "well-behaved", so let's look at another dataset which is a little different.
Let's run \texttt{fastqc} \& inspect an RNA-Seq dataset.

\begin{lstlisting}
cd ~/rawData/small_RNA
mkdir -p ~/QC/small_RNA/rawData
fastqc -o ~/QC/small_RNA/rawData -t 2 SmallRNA_SRR2039540.fastq.gz
cd ~/QC/small_RNA/rawData
firefox SmallRNA_SRR2039540_fastqc.html &
\end{lstlisting}

The RNAseq dataset that we will run \texttt{fastqc} on is a small-RNA dataset, the library containing all the small RNA sequences expressed by the sample at one time-point. This can contain a number of different types of RNA molecules, such as:
\begin{itemize}
\item microRNA (miRNA)
\item piwi-interacting RNA (piRNA)
\item small interfering RNA (siRNA)
\item small nucleolar RNA (snoRNAs)
\item tRNA-derived small RNA (tsRNA)
\item small rDNA-derived RNA (srRNA)
\item small nuclear RNA, also commonly referred to as U-RNA
\end{itemize}

All these RNA types are less than 200bp, however the majority are likely to be microRNAs and piRNAs that are less than 30bp.

\begin{questions}
What is the read length of this sequencing run? \\
\begin{answer}
50bp \\
\end{answer}
What sort of biases or contamination can we expect with this small RNA sequencing run, given that the DNA inserts are likely to be $>30$bp? \\
\begin{answer}
As the DNA insert length is less than the sequencing read length, we should expect that there will be significant adapter contamination. This will need to be trimmed
\end{answer}
\end{questions}

\paragraph{Overrepresented Sequences}
Go to the \textit{over-represented} sequences section for SmallRNA\_SRR2039540.fastq.gz.
There seem to be a large number of over-represented sequences which seem to contain Illumina Single End Adapter 1. These sequences look to be almost $>26$\% of all sequences! You will also notice that all the other over-represented sequences are similar to Illumina Single End Adapter 1 and only differ by a few SNPs. The top 5 over-represented sequences looks extremely similar: \\

\begin{lstlisting}
TGAGATCGTTCAGTACGGCAATCGTATGCCGTCTTCTGCTTGAAAAAAA
TGAGATCGTTCAGTACGGCAATTCGTATGCCGTCTTCTGCTTGAAAAAA
TGACTAGAGACACATTCAGCTTCGTATGCCGTCTTCTGCTTGAAAAAAA
CACCCGTACATATGTTTCCGTGCTTCGTATGCCGTCTTCTGCTTGAAAA
\end{lstlisting}

Most of the sequences show a similar pattern, starting with "TGAG" type sequence and finishing with a "TCTTCTGCTTG" motif and a odd poly-adenine stretch of bases. Each of these sequences are also 49-50bp, exactly the same as our read length.

\paragraph{Adapter Contamination}
Curiously however when we look at the \textit{Adapter Content} section of the report we see no sign of significant adapter contamination.
By the time you reach the 60th nucleotide in the reads from Sample1, this plot tells you that $>50$\% of reads contain matches to the Illumina Universal Adapter.

\paragraph{Per base GC content}
Looking at the "Per base GC content" and there seems to be a massive increase in sequences at a very specific GC content. While the peak number of reads from theoretical distribution of the dataset should be around 20,000 reads $>43$\%, the number of reads at that peak has more than doubled

\paragraph{Per base sequence content}
If you look at the "Per base sequence content" section, this plot looks all over the place! Across the X-axis (base-pair position on the read) we see massive spikes in each nucleotide content. Starting from base-pair 1, Thymine looks to be found in $>70$\% of the sequences, followed by Guanine in base 2 and Adenine for base 3. This is quite consistent with our over-represented sequences above. \\
Additionally, at the end of the read, we see that Adenine is the majority of all base's from 41-50bp, also consistent with our over-represented sequences.

\begin{questions}
What is the explaination of the spike in the number of reads at $>43$\% and the $>60$\% of each nucleotide across the read? \\
\begin{answer}
This is due to one particular sequence dominating the library.
\end{answer}
What possible library preparation issue has caused the pattern seen in this FastQC report?
\begin{answer}
The mostly likely explanation for the dominance of one particular sequence (that happens to be a significant match to the Illumina Single End PCR Adapter), is that during amplification of the library, the library primers amplified an empty DNA insert and therefore the only nucleotides that were sequenced were the 3' adapter.
This is what we call a "primer dimer"
\end{answer}
\end{questions}


\begin{information}
Interpreting the various sections of the report can take time \& experience.
A description of each of the sections is available from the \texttt{fastqc} authors at \url{http://www.bioinformatics.babraham.ac.uk/projects/fastqc/Help/}
\end{information}

\begin{bonus}
Another interesting report is available at \url{http://www.bioinformatics.babraham.ac.uk/projects/fastqc/RNA-Seq\_fastqc.html}
Whilst the quality scores generally look pretty good for this one, see if you can find a point of interest in this data.
This is a good example, of why just skimming the first plot may not be such a good idea.
\end{bonus}

\begin{advanced}
In our dataset of two samples it is quite easy to think about the whole experiment \& assess the overall quality.
What about if we had 100 samples?
Each .zip archive contains text files with the information which can easily be parsed into an overall summary. \\

Whilst this will require low-level scripting skills to perform on an experiment, we can quickly look at two of the important files today.
The overall summary in terms of PASS/FAIL is contained in the `summary.txt' file within the archive.
Open this file in the \texttt{less} pager, and once you've had a look type \texttt{q} to quit, as we have become familiar with.
First make sure you are in the correct directory, then inspect these files using \texttt{less}.
\begin{lstlisting}
cd ~/QC/RNASeq/rawData
unzip -oc SmallRNA_SRR2039540_fastqc.zip '*summary.txt' | less
\end{lstlisting}

The raw numbers for each of the sections are in the file fastqc\_data.txt.
Page through the file, until you lose interest then quit the pager.
\begin{lstlisting}
unzip -oc SmallRNA_SRR2039540_fastqc.zip '*fastqc_data.txt' | less
\end{lstlisting}

We can also extract any specific image file for compiling into a pdf, or find whatever we need by using these ideas.
This makes handling the data for a large experiment much simpler.
There are plenty of hints online for how to write a \textit{shell script}, or alternatively, attend one of our scripting workshops.
\end{advanced}

\section{Further Reading}
An excellent article which deals with some of the issues around data quality is:

Zhou, X and Rokas, A. (2014). \textit{Prevention, diagnosis and treatment of high-throughput sequencing data pathologies.} Molecular Ecology 23, 1679-1700.

This has been included on your VM as the file QC.pdf \& contains many examples of good data and low quality data, as well as a detailed discussion.
If you feel like you are running ahead of schedule, or if you finish early it may be a good opportunity to download \& read through the article.
The workflow given at the end may also be particularly useful.
