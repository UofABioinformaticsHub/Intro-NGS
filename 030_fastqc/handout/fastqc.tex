\setModuleTitle{FASTQC}
\setModuleAuthors{%
    Steve Pederson, Bioinformatics Hub, University of Adelaide \mailto{stephen.pederson@adelaide.edu.au}\\
}
\setModuleContributions{%
  Dan Kortschak, Adelson Research Group, University of Adelaide \mailto{dan.kortschak@adelaide.edu.au} \\
  Terry Bertozzi, SA Museum \mailto{Terry.Bertozzi@samuseum.sa.gov.au}
}

\chapter{\moduleTitle}
\newpage

\section{Using fastqc}
\begin{steps}
Removal of low-confidence base calls is an important part of any NGS analysis, and we can begin this process by checking the quality of libraries using the tool \texttt{fastqc}.
As with all programs on the command line, we need to see how it works before we use it.
The following command will open the help file in the \texttt{less} pager which we used earlier.
To navigate through the file, use the \textless spacebar\textgreater ~to move forward a page, \textless \texttt{b}\textgreater ~to move back a page \& \textless \texttt{q}\textgreater ~to exit the manual. \\
\begin{lstlisting}
fastqc -h | less
\end{lstlisting}
\end{steps}

\begin{note}
Fastqc will create an html report on each file you submit, which can be opened from any web browser, such as \texttt{firefox}
As seen in the help page, \texttt{fastqc} can be run from the command line or from a graphic user interface (GUI).
Using a GUI is generally intuitive, so today we will look at the command line usage, as that will give you more flexibility \& options going forward.
Some important options for the command can be seen in the manual.\\
\end{note}
\begin{steps}
As you will see in the manual, setting the \texttt{-h} option as above will call the help page.
Look up the following options to find what they mean. \\
\begin{center}
\begin{tabular}[h]{|p{4cm}|p{8cm}|}
  \hline
  \textbf{Option} & \textbf{Usage} \\
  \hline
  -o & \\
   & \\
   \hline
  -t & \\
   & \\
   \hline
   -{}-casava & \\
   & \\
   \hline
\end{tabular}
\end{center}
\end{steps}

\begin{steps}
As we have two RNA-Seq files, we will first need to create the output directory, then we can run \texttt{fastqc} using 2 threads which will ensure the files are processed in parallel.
This can be much quicker when dealing with large experiments.
\begin{lstlisting}
cd ~/NGSData/RNASeq/rawData
mkdir -p ~/QC/RNASeq/rawData
fastqc -o ~/QC/RNASeq/rawData -t 2 reads1.fq.gz reads2.fq.gz
\end{lstlisting}
It's probably a good idea to scribble a note next to each line if you didn't understand what you did.
If you haven't seen the command \texttt{mkdir} before, check the help page 
\begin{lstlisting}
man mkdir
\end{lstlisting}
\end{steps}

\begin{steps}
The above command gave both files to fastqc, told it where to write the output (\texttt{-o \~{}/QC}) \& requested two threads (\texttt{-t 2}). 
The reports are in the html files, with all of the plots stored in the zip files. 
To look at the QC report for each file, we can use \texttt{firefox}.
\begin{lstlisting}
cd ~/QC/RNASeq/rawData
ls -lh
firefox reads1.fq_fastqc.html reads2.fq_fastqc.html &
\end{lstlisting}
The left hand menu contains a series of click-able links to navigate through the report, with a quick guideline about each section given as a tick, cross of exclamation mark.
\end{steps}

\begin{note}
Two hints which may make your inspection of these files easier are:
\begin{enumerate}
	\item To zoom out in \texttt{firefox} use the shortcut Ctrl-. Reset using Ctrl0 and zoom in using Ctrl+
	\item You can open these directly from a traditional directory view by double clicking on the .html file.
\end{enumerate}
If your terminal seems busy after you close \texttt{firefox}, use the 'Ctrl C' shortcut to stop whatever is keeping it busy
\end{note}

\begin{questions}
How many reads are there in both files?\\
\begin{answer}
  2500000 \\
\end{answer}
How long are the reads in these files?\\
\begin{answer}
  37bp \\
\end{answer}
\end{questions}

\section{Interpreting the FASTQC Report}
\begin{note}
As we work through the QC reports we will develop a series of criteria for filtering and cleaning up our files.
There is usually no perfect solution, we just have to make the best decisions we can based on the information we have.
Some sections will prove more informative than others, and some will only be helpful if we are drilling deeply into our data.
Firstly we'll just look at a selection of the plots.
We'll investigate some of the others with some `bad' data later.
\end{note}

\subsubsection*{Per Base Sequence Quality}
\begin{steps}
Both of the files should be open in \texttt{firefox} in separate tabs.
Perform the following steps on both files.
Click on the \texttt{Per base sequence quality} hyperlink on the left of the page \& you will see a boxplot of the QC score distributions for every position in the read.
This is the main plot that researchers will look at for making informed decisions about later stages of the analysis.
\end{steps}

\begin{questions}
What do you notice about the QC scores as you progress through the read? \\
\begin{answer}
They clearly drop off as the read extends.\\
\end{answer} 
\end{questions}

We will deal with trimming the reads in a later section, but start to think about what you should do to these reads to ensure the highest quality in your final alignment \& analysis.

\paragraph{Per Tile Sequence Quality}
This section just gives a quick visualisation about any physical effects on sequence quality due to the tile within the each flowcell.
For the first file, you will notice an even breakdown in the quality of sequences near the end of the reads across all tiles.
In our second QC report, you will notice a poor quality around the 25th base in the 2nd (or 3rd) tile.
Generally, this would only be of note if drilling deeply to remove data from tiles with notable problems.
Most of the time we don't factor in spatial effects, unless alternative approaches fail to address the issues we are dealing with.

\paragraph*{Per Sequence Quality Scores}
This is just the distribution of average quality scores for each sequence.
There's not much of note for us to see here.

\paragraph{Per Base Sequence Content}
This will often show artefacts from barcode sequences or adapters early in the reads, before stabilising to show a relatively even distribution of the bases.

\paragraph{Sequence Duplication Levels}
This plot shows about what you'd expect from an RNA-Seq experiment.
There are a few duplicated sequences (rRNA?, highly expressed genes? etc.) and lots of unique sequences represented the diverse transcriptome.
This is only calculated on a small sample of the library for computational efficiency and is just to give a rough guide if anything unusual stands out

\paragraph{Kmer Content}
Statistically over-represented sequences can be seen here \& often they will overlap. 
In our first plot, the some sequences derive from the same motif, and are the extracts of the same longer sequence, just shifted along one base.
No information is given as the source of these sequences, and you would expect to see barcode sequences or motifs that correspond to any digestion protocols here.
This is a plot that can cause significant confusion, but can alert you to any unexpected sequence-based problems in your data.

\subsection{Some Flawed Data}
This RNA-Seq dataset is relatively ``well-behaved", so let's look at another dataset which is a bit more problematic.
First we'll need to create the directory structure and download the files.

\begin{lstlisting}
cd
mkdir -p NGSData/iCLIP/rawData
cd NGSData/iCLIP
wget https://swift.rc.nectar.org.au:8888/v1/AUTH_33065ff5c34a4652aa2fefb292b3195a/IntroNGS_2015/iCLIP_adapters.txt
cd rawData
wget https://swift.rc.nectar.org.au:8888/v1/AUTH_33065ff5c34a4652aa2fefb292b3195a/IntroNGS_2015/iCLIP.zip
unzip iCLIP.zip && rm iCLIP.zip
\end{lstlisting}

Let's run \texttt{fastqc} \& inspect the iCLIP dataset, which is a version of RNA-Seq data where the RNA has been pulled down via IP after cross-linking to an RNA-associated protein (Argonaute).

\begin{lstlisting}
mkdir -p ~/QC/iCLIP/rawData
cd ~/NGSData/iCLIP/rawData
fastqc -o ~/QC/iCLIP/rawData -t 2 iCLIP_Sample1.fastq iCLIP_Sample2.fastq
cd ~/QC/iCLIP/rawData
firefox iCLIP_Sample1_fastqc.html iCLIP_Sample2_fastqc.html &
\end{lstlisting}

There are several things to note about these reports.
Firstly, we know from inspection that these fastq libraries contain reads which have failed the Illumina Chastity filter.
We would expect these to show up in the \textit{Basic Statistics} table, but they don't by default.

\begin{questions}
How would we know that there are low quality reads here if the Illumina flag is ignored?\\
\begin{answer}
The overall lower distribution of the quality scores should raise a red flag to us. \\
\end{answer}
\end{questions}

To see how these reads look without the flagged reads, we can turn on the \texttt{-{}-casava} option when running \texttt{fastqc}.
You can ignore any warning messages you receive about not looking like part of a CASAVA group.

\begin{lstlisting}
cd ~/NGSData/iCLIP/rawData
fastqc --casava -o ~/QC/iCLIP/rawData -t 2 iCLIP_Sample1.fastq iCLIP_Sample2.fastq
cd ~/QC/iCLIP/rawData
firefox iCLIP_Sample1_fastqc.html iCLIP_Sample2_fastqc.html &
\end{lstlisting}

\begin{information}
Note that the above code will have over-written the original QC reports.
As this second analysis is the best approach, this is actually OK at this point.
However. this can be a common pitfall when running an analysis, which is why we will often record our analysis as a script.
That way we have a record of what we have done and know exactly what options we have set for a process.
\end{information}


\paragraph{Adapter Contamination}
Go to the over-represented sequences section for Sample1.
There seem to be a large number of over-represented sequences here which are not listed as matching any known sequences.
Scanning through these manually is near impossible and solving this may take some leg work.
Shift over to the same section in Sample2 and you will notice about 23\% of sequences seem to contain an Illumina Paired End PCR Primer.
The next line also contains matches to this but with some indels. \\

Now head to the \textit{Adapter Content} section of each report \& consider what we are seeing here.
By the time you reach the 60th nucleotide in the reads from Sample1, this plot tells you that $>50$\% of reads contain matches to the Illumina Universal Adapter.


\subsection{Some More Example Reports}
We'll actually try to clean the iCLIP dataset up in the next chapter, but for now let's just head to another sample plot at \url{http://www.bioinformatics.babraham.ac.uk/projects/fastqc/bad\_sequence\_fastqc.html}


\paragraph{Per Base Sequence Quality}
Looking at the first plot, we can clearly see this data is not as high quality as the one we have been exploring ourselves.

\begin{questions}
Consider that the minimum sequence length required for confidently obtaining unique alignments is >20bp.
Two approaches to this data might be to only include high quality sequences, or to trim the low quality bases from the ends and use shorter reads for downstream analysis.
What would be the consequences of either approach? \\
\begin{answer}
If we excluded low quality sequences, we would throw away a large amount of data.
Depending on the question we are asking of the data, this may render our experiment meaningless or may help us find more accurate results.
Context is everything. \\
If we trimmed the reads, we may retain a larger number of them, but more may map to non-unique locations in the reference.
Once again, context is everything.\\
\end{answer}
\end{questions}

\paragraph{Per Tile Sequence Quality}
Some physical artefacts are visible \& some tiles seem to be consistently lower quality.
Whichever approach we take to cleaning the data will more than likely account for any of these artefacts.
Sometimes it's just helpful to know where a problem has arisen.

\paragraph{Overrepresented Sequences}
Head to this section of the report \& scan down the list.
Unlike our sample data, there seem to be a lot of enriched sequences of unknown origin.
There is one hit to an Illumina adaptor sequence, so we know at least one of the contaminants in the data.
Note that some of these sequences are the same as others on the list, just shifted one or two base pairs.
A possible source of this may have been non random fragmentation.

\paragraph{Kmer Content}
\begin{questions}
Do you notice anything unusual about this plot?\\
\begin{answer}
The K-mers are present at the end of the reads.
Was this a problem with sample preparation? 
Do these map to barcodes, adaptors or primers at the other end of the reads? \\
\end{answer}
\end{questions}

\begin{information}
Interpreting the various sections of the report can take time \& experience.
A description of each of the sections is available from the \texttt{fastqc} authors at \url{http://www.bioinformatics.babraham.ac.uk/projects/fastqc/Help/}
\end{information}

\begin{bonus}
Another interesting report is available at \url{http://www.bioinformatics.babraham.ac.uk/projects/fastqc/RNA-Seq\_fastqc.html}
Whilst the quality scores generally look pretty good for this one, see if you can find a point of interest in this data.
This is a good example, of why just skimming the first plot may not be such a good idea.
\end{bonus}

\begin{advanced}
In our dataset of two samples it is quite easy to think about the whole experiment \& assess the overall quality.
What about if we had 100 samples? 
Each .zip archive contains text files with the information which can easily be parsed into an overall summary. \\

Whilst this will require low-level scripting skills to perform on an experiment, we can quickly look at two of the important files today.
The overall summary in terms of PASS/FAIL is contained in the `summary.txt' file within the archive.
Open this file in the \texttt{less} pager, and once you've had a look type \texttt{q} to quit, as we have become familiar with.
First make sure you are in the correct directory, then inspect these files using \texttt{less}.
\begin{lstlisting}
cd ~/QC/RNASeq/rawData
unzip -oc reads1.fq_fastqc.zip '*summary.txt' | less
\end{lstlisting}

The raw numbers for each of the sections are in the file fastqc\_data.txt.
Page through the file, until you lose interest then quit the pager.
\begin{lstlisting}
unzip -oc reads1.fq_fastqc.zip '*fastqc_data.txt' | less
\end{lstlisting}

We can also extract any specific image file for compiling into a pdf, or find whatever we need by using these ideas.
This makes handling the data for a large experiment much simpler.
There are plenty of hints online for how to write a \textit{shell script}, or alternatively, attend one of our scripting workshops.
\end{advanced}

\section{Further Reading}
An excellent article which deals with some of the issues around data quality is:

Zhou, X and Rokas, A. (2014). \textit{Prevention, diagnosis and treatment of high-throughput sequencing data pathologies.} Molecular Ecology 23, 1679-1700.

This has been included on your VM as the file QC.pdf \& contains many examples of good data and low quality data, as well as a detailed discussion.
If you feel like you are running ahead of schedule, or if you finish early it may be a good opportunity to download \& read through the article.
The workflow given at the end may also be particularly useful.