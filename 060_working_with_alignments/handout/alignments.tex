\setModuleTitle{Working With Alignments}
\setModuleAuthors{%
    Steve Pederson, Bioinformatics Hub, University of Adelaide \mailto{stephen.pederson@adelaide.edu.au}\\
}
\setModuleContributions{%
  Dan Kortschak, Adelson Research Group, University of Adelaide \mailto{dan.kortschak@adelaide.edu.au} \\
  Terry Bertozzi, SA Museum \mailto{Terry.Bertozzi@samuseum.sa.gov.au}
}

\chapter{\moduleTitle}
\newpage

\section{The SAM/BAM file format}
\begin{information}
Reads that have been aligned to a reference are no longer stored in fastq format but are stored in either SAM or BAM format.
These two formats are virtually identical, however the SAM format is a text file which is easily readable to human eyes, whilst a BAM file is the same information converted to binary.
This conversion means that file sizes are smaller, and that computational processes can be performed more efficiently.
Typically, we work with BAM files as these provide considerable gains in storage space \& analytic speed.
The tools we use to inspect these files are provided in the package samtools, which has been installed on your VM. \\
\end{information}

\begin{steps}
The reads from the previous dataset which mapped to \textit{chr2L} of \textit{D. melanogaster} are in the folder \texttt{\~{}/NGSData/RNASeq/alignedData}.
Let's look at the first 5 lines of the file.
\begin{lstlisting}
cd ~/NGSData/RNASeq/alignedData
head -n5 chr2L.sam  
\end{lstlisting}
\end{steps}

\begin{questions}
Can you distinguish between the header of the SAM format and the actual alignments?
\begin{answer}
The header line starts with the letter `@', i.e.: \\
\begin{tabular}{llll}
@HD & VN:1.0 & SO:unsorted & \\
@SQ & SN:chr2L & LN:23011544 & \\
@PG & ID:TopHat	& VN:1.2.0 & CL:/g/steinmetz/collaboration/... \\
\end{tabular}

While the actual alignments start with read id, i.e.: \\

\begin{tabular}{llll}
SRR031714.5049824 & 99 & chr2L & etc \\
SRR031714.5049824 & 147 & chr2L & etc \\
\end{tabular}

\end{answer}
What kind of information does the header provide? \\
\begin{answer}
\begin{itemize}
\item @HD: Header line; VN: Format version; SO: the sort order of alignments.
\item @SQ: Reference sequence information; SN: reference sequence name; LN: reference sequence length.
\item @PG: Read group information; ID: Read group identifier; VN: Program version; CL: the command line that produced the alignments.
\end{itemize}
\end{answer}
\end{questions}

\section{Conversion to BAM format}
\begin{steps}
The BAM format is much more convenient computationally, so we can convert this file into the BAM format using \texttt{samtools view}.
Whilst this line of code may look counter-intuitive, remember that the first section is all the information about how to process the input, \& specify the output, whilst the final section is the file that the command will operate on.
First look through the help page to understand what you are asking the tool to perform. \\
\begin{lstlisting}
samtools view 
samtools view -bS -o chr2L.bam chr2L.sam
\end{lstlisting}
Have a look at the file sizes to see the difference.
Note that the BAM file is about 12\% of the size, which can be an important consideration when managing large datasets.
\begin{lstlisting}
ls -lh
\end{lstlisting}
\end{steps}

\section{Handling BAM files}
\begin{steps}
As BAM files are in binary format they will look like gibberish if we try to read them directly.
Instead we can inspect them by using \texttt{samtools view} again.
To view the header we use:\\
\begin{lstlisting}
samtools view -H chr2L.bam
\end{lstlisting}
Whereas to look through the first few mapped reads, we can use:
\begin{lstlisting}
samtools view chr2L.bam | head
\end{lstlisting}
As this data can easily spill across lines, it might be helpful to maximise your terminal to see the complete line structure.
\end{steps}

\begin{questions}
Why didn't we need to set the \texttt{-b} or \texttt{-S} option when running \texttt{samtools view} in the previous two lines?\\
\begin{answer}
These option is only used when specifying the output as a BAM file (-b) and the input as a SAM file (-S).\\
\end{answer}
\end{questions}

\subsection{The SAM/BAM data structure}
\begin{information}
If we understand what information is contained within a file, we can know what decisions to make as we progress with our analysis, so let's have a look at what the data structure is for a SAM/BAM file.
A SAM file is \textit{tab-delimited}, which means that each field is separated by a tab, giving a data structure effectively consisting of columns (or fields).
In order, these are: \\
\begin{tabular}{r l l}
1 & QNAME & Query template/pair NAME \\
2 & FLAG & bitwise FLAG \\
3 & RNAME & Reference sequence NAME \\
4 & POS & 1-based leftmost POSition/coordinate of clipped sequence \\
5 & MAPQ & MAPping Quality (Phred-scaled) \\
6 & CIGAR & extended CIGAR string \\
7 & MRNM & Mate Reference sequence NaMe (`=' if same as RNAME) \\
8 & MPOS & 1-based Mate POSistion \\
9 & TLEN & inferred Template LENgth (insert size) \\
10 & SEQ & query SEQuence on the same strand as the reference \\
11 & QUAL & query QUALity (ASCII-33 gives the Phred base quality) \\
12+ & OPT & variable OPTional fields in the format TAG:VTYPE:VALUE \\
\end{tabular}
\end{information}

\begin{note}
Several of these fields contain useful information, so looking the the first few lines which we displayed above, you can see that these reads are mapped in pairs as consecutive entries in the QNAME field are often (but not always) identical.
Most of these fields are self-explanatory, but some require exploration in more detail.
\end{note}

\subsection{SAM Flags}
\begin{information}
These are quite useful pieces of information, but can be difficult at first look.
Head to \url{http://broadinstitute.github.io/picard/explain-flags.html} to see a helpful description.
The simplest way to understand these is that it is a bitwise system so that each description heading down the page increases in a binary fashion.
The first has value 1, the second has value 2, the third has value 4 \& so on until you reach the final value of 2048.
The integer value contained in this file is the unique sum of whichever attributes the mapping has.
For example, if the read is paired \& mapped in a proper pair, but no other attributes are set, the flag field would contain the value 3.
\end{information}

\begin{questions}
What value could a flag take if it had the following set of properties?
	\begin{enumerate}
	\item the read was paired; 
	\item the read was mapped in a proper pair
	\item the read was was the first in the pair; \& 
	\item the alignment was a supplementary alignment
	\end{enumerate}
\begin{answer}
2115 \\
\end{answer}
Some common values in the bam file are 99, 147 \& 145.
Look up the meanings of these values. \\
\end{questions}

\begin{information}
Things can easily begin to confuse people once you start searching for specific flags, but if you remember that each attribute is like an individual flag that is either on or off (i.e. it is binary).
If you searched for flags with the value 1, you wouldn't obtain the alignments with the exact value 1, rather you would obtain the alignments for which the first flag is set \& these can take a range of values.
\end{information}

\begin{steps}
Let's try this using the command \texttt{samtools view} with the option \texttt{-f N} to include reads with a flag set and the option \texttt{F N} to exclude reads with a specific flag set.
Let's get the first few reads which are mapped in a proper pair, so the flag `2' will be set.\\
\begin{lstlisting}
samtools view -f 2 chr2L.bam | head
\end{lstlisting}
Note that none of the flags actually have the value 2, but if you typed the values 99, 147 or 163 into the web page, you'll see that this flag is set for all of these values.
Similarly if we wanted to extract only the reads which are NOT mapped in a proper pair we would change the option to a upper-case F.
\begin{lstlisting}
samtools view -F 2 chr2L.bam | head
\end{lstlisting}
Again, try entering a few of these sample values into the web page and you will see that this flag is not set for any of these values.
\end{steps}

\begin{information}
This can be a very helpful tool for extract subsets of your aligned reads.
For example, we can create a new BAM file with only the reads which were aligned in a proper pair by entering the following command.
\begin{lstlisting}
samtools view -f 2 -bo chr2L.paired.bam chr2L.bam
ls -lh
\end{lstlisting}
You can pull out highly specific combinations of alignments should you so choose
\end{information}


\subsection{CIGAR strings}
\begin{note}
These give useful information about the type of alignment that has been performed on the read.
In the first few reads we called up earlier, most had the value \texttt{37M}, which stands for 37 \textbf{M}atches.
The other abbreviations in common use are I (insertion) \& D (deletion), however these are not found in this bam file.
\end{note}

\begin{questions}
A CIGAR string contained in this file is `10M240N27M'.
Can you think of an interpretation of this string? \\
\begin{answer}
10 Matches, 240 Skipped bases in the reference, then 27 Matches.
This may represent a splice junction as determined by \texttt{tophat}
\end{answer}
\end{questions}

\begin{advanced}
Knowing this pattern as splice junctions, you could write a \textit{regular expression} search to find only the alignments that contain potential splice junctions.
\begin{questions}
What would be a pattern to search for, and how would you create this as a new BAM file.
\begin{answer}
This will pull out the first few.
\begin{lstlisting}
samtools view chr2L.bam | egrep '[0-9]+M[0-9]+N[0-9]+M' | head
\end{lstlisting}
To really do it correctly, you'd probably have to fudge a new sam file with the header, the cat the output \& convert to BAM.
\end{answer}
\end{questions}
\end{advanced}

\section{Viewing the Alignments}
In this final section we will use the package IGV Viewer to inspect the alignments in this bam file.
We are going to use the IGV genome browser, which has been installed on your VM.
This is just a simple tool for manually inspecting your alignments, and getting a feel for what levels of coverage you have in your data, or if there are any strange artefacts.

\subsection{Sorting and Indexing Our BAM Files}
\begin{note}
To visualise our alignments, we first need to sort our bam file so that all the alignments are in order based on each chromosome, instead of the order they appeared in the source fastq file.
This allows much faster searching of the files, and also allows for better compression.
The second step will be to index the sorted file to enable faster file access by the browser.
\end{note}

\begin{steps}
\begin{lstlisting}
cd ~/NGSData/RNASeq/alignedData
samtools sort chr2L.bam chr2L.sorted
samtools index chr2L.sorted.bam
\end{lstlisting}
\end{steps}

\begin{information}
Note that in keeping with the weirdness of bioinformatics \textit{the input file is not specified last in the tool} \texttt{samtools sort}.
This would normally be the convention but for some reason the programmers have chosen not to follow this convention.
The final setting given to the command is the prefix that the output file will be given. \\
\end{information}

The command \texttt{samtools index} then created a file with the exact same name as the input file, but with the extra suffix \texttt{.bai}, which indicates this is an index file.\\

\begin{steps}
Now that we've got our files ready to view we can load the IGV browser.
This may take a minute or two so please be patient.

\begin{lstlisting}
igv
\end{lstlisting}
\end{steps}

\begin{steps}
Once the browser has opened, go to the drop-down menu \& load the genome for \textit{Drosophila melanogaster} (dm3).
(This will create a .genome file in your \texttt{\~{}/igv} folder.)
Now we can load our sorted .bam file using  \textit{File $>$ Load from File} in the top menu, and make sure you load the sorted .bam file.
This will automatically check for the indexed file we created a minute ago. \\

In the second ``button" go to the region chr2L, then enter the word \textit{poe} in the window next to this.
This will take us to the gene \textit{poe} and we will be able to see the reads in our .bam file that have been aligned to this gene.
Unfortunately with these alignments, the further out we zoom we will begin to lose our alignment information.

Have a look at the gene \textit{drongo} as well, and have a look around by clicking the navigation at the very top of the graphical display.
\end{steps}

\section{Bonus For Those Well Ahead of the Pack}
\begin{advanced}
For those who still feel like they want more, try running an entire pipeline on the iCLIP data, including adapter removal and read trimming.
Record your commands in the \texttt{gedit} text editor so you have a record of what you did.
Instead of aligning to the entire human genome, try just aligning to chromosome 17.
\end{advanced}


