\setModuleTitle{Aligning Reads To a Reference Sequence}
\setModuleAuthors{%
    Steve Pederson, Bioinformatics Hub, University of Adelaide \mailto{stephen.pederson@adelaide.edu.au}\\
}
\setModuleContributions{%
  Dan Kortschak, Adelson Research Group, University of Adelaide \mailto{dan.kortschak@adelaide.edu.au} \\
  Terry Bertozzi, SA Museum \mailto{Terry.Bertozzi@samuseum.sa.gov.au}
}

\chapter{\moduleTitle}
\newpage

%% Do an alignment just to one chromosome

\begin{note}
Once we have cleaned our data of any contaminating sequences, and removed the bases which are more likely to contain errors we can more confidently align our reads to a reference.
Different experiments may have different reference sequences depending on the context.
For example, if we have a sub-sample of the genome associated with restriction sites like RAD-Seq, we would probably align to a reference genome, or if we have RNA-Seq we might choose to align to the transcriptome instead of the whole genome.
Alternatively, we might be interested in \textit{de novo} genome assembly where we have no reference genome to compare our data to.
\end{note}

\begin{questions}
For our RNA-Seq data, we could align to the genome or the transcriptome.
What might be some of the disadvantages of either approach? \\
\begin{answer}
This is more of a discussion question for people to mull over.
If aligning to the genome, we have the potential to discover novel transcribed regions, however dealing with alternate transcripts can still be difficult.
Mapping to the transcriptome may be relatively easier \& faster due to the contiguous mappings, however alternate isoforms can still present difficulties.
Novel isoforms will also fail to be detected using this approach, or undocumented intergenic RNAs. \\
\end{answer}
\end{questions}

\section{How Aligning Works}
\begin{note}
Most fast aligners in widespread public use are based on a technique called the \textit{Burrows-Wheeler Transform}, which is essentially a way of restructuring, or indexing, the genome to allow very rapid searching.
This technique comes from computer science \& is really beyond the scope of what most of us need to know.
The essence of it is that we have a very fast searching method, and most aligners use a seed sequence within each read to begin the searching.
These seeds are then expanded outwards to give the best mapping to a sequence.
There are many different alignment tools available today \& each one will have a particular strength.
For example, \texttt{bowtie} is very good for mapping short reads like the ones we have in our RNASeq dataset, whilst \texttt{bowtie2} is more suited to reads longer than 50bp.
\end{note}

\subsubsection{What's the difference}
Some key differences between aligners is in the way they index the genome, and in the way they are equipped to handle mismatches \& indels.
Choosing an aligner can be a difficult decision with the differences often being quite subtle.
Sometimes there is a best choice, other times there really isn't.
Make sure you've researched relatively thoroughly before deciding which to use.

\section{Aligning our RNASeq reads}
\subsubsection{Downloading A Reference Genome}
\begin{steps}
To align any reads, we first need to download the appropriate (i.e. latest) genome \& then we can build the index to enable fast searching via the Burrows-Wheeler Transform.
The RNASeq reads we have today come from \textit{Drosophila melanogaster}, so to find the genome, open Firefox \& head to \url{ftp://ftp.ensembl.org/pub/release-62/fasta/drosophila\_melanogaster}.
Here you can find the transcriptome (\textit{cdna}), genome (\textit{dna}), proteome (\textit{pep}) \& non-coding RNA (\textit{ncrna}) in separate folders.
Let's align to the genome, so first create the directory to save it to then download using the linux command \texttt{wget}.
Instead of typing that enormous last command, navigate to the /dna folder on the ftp server then right-click on the file Drosophila\_melanogaster.BDGP5.25.62.dna\_rm.toplevel.fa.gz \& select \texttt{Copy Link Location}.
After typing \texttt{wget} in the following commands, right-click in the terminal \& paste the contents of the clipboard.
(An alternative method for pasting into the terminal is to use \texttt{Ctrl + Shift + V})
\begin{lstlisting}
cd ~
mkdir reference
cd reference
wget 'Paste Copied Link Location Here'
gunzip Drosophila_melanogaster.BDGP5.25.62.dna_rm.toplevel.fa.gz
\end{lstlisting}
\end{steps}

\begin{information}
Now we have unzipped the Drosophila genome, we should have a quick look at the file.
It will contain all chromosomes \& the mitochondrial sequences.
\end{information}

\begin{steps}
We can print out the first 10 lines using the \texttt{head} command.
\begin{lstlisting}
head Drosophila_melanogaster.BDGP5.25.62.dna_rm.toplevel.fa    
\end{lstlisting}
Note that the first line describes the following sequence \& begins with a \textgreater ~symbol.
We can use this to search within the file using \textit{regular expressions} \& print all of these description lines.
\begin{lstlisting}
grep '^>' Drosophila_melanogaster.BDGP5.25.62.dna_rm.toplevel.fa
\end{lstlisting}
Alternatively could simply count them using the \texttt{-c} option.
\begin{lstlisting}
grep -c '^>' Drosophila_melanogaster.BDGP5.25.62.dna_rm.toplevel.fa  
\end{lstlisting}
\end{steps}

\begin{questions}
What does the ``rm'' stand for in the filename?\\
\begin{answer}
Repeat Masker. This will have repeats \& other difficult sequences masked.
\end{answer}
\end{questions}

\subsubsection{Building an Index}
We will align using the tool \texttt{tophat} which is based on the aligner \texttt{bowtie} so requires an index to be built using \texttt{bowtie-build}. 
This tool is specially designed to handle splice junctions \& is highly suitable for aligning RNASeq reads to a reference genome, such as we have\\

\begin{steps}
Once again, we need to check the help pages.
Unfortunately the \texttt{tophat} page isn't very friendly on the screen so after entering the following you might have to scroll up a little.
There are a wealth of options available for you to look through.\\
\begin{lstlisting}
tophat -h
\end{lstlisting}

We should also inspect the help page for \texttt{bowtie-build} which we will use to build the index.\\
\begin{lstlisting}
bowtie-build -h
\end{lstlisting}
Note that in the output from this process we are not going to specify a file, but rather a \texttt{basename} which will be included in all files.
\end{steps}

\begin{steps}
Now that we've had a look, type to following command which will take a few minutes to run, so don't worry if it looks like nothing is happening.
\begin{lstlisting}
bowtie-build Drosophila_melanogaster.BDGP5.25.62.dna_rm.toplevel.fa D_melanogaster.BDGP5.25.62.dna_rm.toplevel
\end{lstlisting}
Note that we could have specified anything as the basename, but keeping the key information which relates to the original file is often good practice, despite it making for long \& cumbersome commands.
This is where using a text editor, with cut \& paste can make life a little easier.
\end{steps}

\begin{steps}
Let's look at what files have been created.
\begin{lstlisting}
ls
\end{lstlisting}
Unfortunately the BWT transformed files are in binary, so we can't really see what they look like, but this is what is required by the aligner \textit{tophat}, as well as \textit{bowtie} 
\end{steps}

\subsection{Aligning the reads}
Unfortunately, this will take hours on our VMs so we can't really perform it in the time we have for the workshop.
Instead, a set of aligned files should be on your VM in the folder \texttt{\~{}/mapped}, which we will explore in the next section.
\textbf{Please do not run this}, but if we wanted to align the trimmed reads, the command would look like the following.
\begin{lstlisting}[style=command_syntax]
tophat D_melanogaster.BDGP5.25.62.dna_rm.toplevel  \ 
 ~/NGSData/RNASeq/trimmedData/reads1_trimmed.fq.gz \
  ~/NGSData/RNASeq/trimmedData/reads2_trimmed.fq.gz
\end{lstlisting}

\begin{advanced}
\begin{questions}
Although we haven't covered the deeper technicalities of aligning reads to the genome today, consider the following scenario:\\

During data generation, an insertion has occurred which makes a read align with an altogether different genomic region.
How can this be corrected so that it aligns to it's original sequence? \\
\begin{answer}
If they can solve that, tell them to write a paper on it \& become outrageously famous...
\end{answer}
\end{questions}
\end{advanced}

\begin{bonus}
If you are running well ahead of time, try downloading the repeat masked version of the sequence for the chromosome 2L from the \texttt{ftp} site.
Align the reads to to this sequence only.
\end{bonus}